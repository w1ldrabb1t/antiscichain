\section{Conclusion}
The Antifragile Science Blockchain platform represents a paradigm shift in how we evaluate and incentivize the generation of scientific knowledge. Instead of relying on human consensus or pre-existing reputations, the ASB platform and protocol is focused on mapping and measuring both the support and refutation evidence across research studies. We discussed how this approach is deeply rooted in the philosophy of science: it embodies Popper's notion of \emph{falsifiability} by structurally rewarding refutation, and it echoes Taleb's insights by making the system antifragile, one that gains from "harm" and "uncertainty".

By contrast, consensus-driven and authority-driven models can too easily conflate popularity with truth. ASB breaks that link, ensuring that even single dissenting experiment contributes with a strong signal when backed by clear evidence. In doing so, it preserves the core scientific virtue of skepticism and continuous improvement. The open, decentralized nature of the protocol further ensures that no viewpoint can be unfairly censored and that the system as a whole learns from its errors over time.

Implementing ASB in practice will mark a significant step toward a more resilient and trustworthy scientific ecosystem. It aligns incentives for researchers to pursue rigorous replication and falsification, knowing that both contribute to improving the overall scientific knowledge base. It provides those who consume scientific information (whether other researchers, policymakers, or the public in general) with a more insightful metric than "citation counts" or a journal's "credibility" — the Evidence Fragility Score, a metric that reflects how fragile a given research is in the arena of evidence testing.

In summary, the ASB platform and its protocol offers a blueprint for a self-correcting, evidence-first approach to evaluating truth and generating knowledge on complex non-linear systems.