\section{Introduction}
Galileo Galilei reputedly remarked that \emph{``in questions of science, the authority of a thousand is not worth the humble reasoning of a single individual.''}\footnote{Often attributed to Galileo's \emph{Dialogue Concerning the Two Chief World Systems} (1632). See, e.g., Arago's \emph{Eulogy of Galileo} (1874) for this quote.} This sentiment underscores that scientific truth is not a matter of majority rule. Yet, many modern mechanisms for evaluating information---from social media upvotes to academic citation counts---implicitly rely on some form of consensus or popularity. If enough people \emph{agree} that a statement is true, it tends to be treated as true. Such approaches carry the risk of elevating well-liked ideas over correct but unpopular ones. History offers plenty of examples where the scientific consensus was later overturned by a single, robust counter-example or a maverick discovery.

In the digital era, the challenges of discerning truth are amplified by information overload and the rapid spread of misinformation. There is growing interest in ``decentralized science'' (DeSci) and blockchain-based platforms for knowledge sharing. Some have proposed using decentralized autonomous organizations (DAOs) or token-weighted voting to assess scientific claims, effectively crowdsourcing credibility. Others rely on traditional reputation systems, assuming that if experts or prestigious institutions endorse a result, it must be reliable. However, consensus-driven filtering can devolve into a popularity contest or groupthink, while reputational gatekeeping can entrench biases and hinder novel insights.

The Antifragile Science Blockchain (ASB) protocol takes a different approach. Instead of polling opinions or deferring to reputations, ASB measures credibility through an \emph{evidence graph} of scientific publications. Each paper or claim is a node, and directed links (citations) carry semantic weight indicating whether the cited result is being supported or challenged. By analyzing this evolving graph of evidence, ASB aims to quantify how well a claim has survived experimental tests over time. In essence, credibility emerges from the \emph{structure of evidence}, not the number of people who believe it.

This paper articulates the epistemological foundations behind ASB and contrasts it with consensus-based and reputation-based approaches to scientific validation. 
In Section-2, we discuss the fragility of knowledge and how easily we are fooled by randomness. 
In Section-3, we contrast the fragility of knowledge with, what we call, Antifragile Learning. Based on Nassim Taleb's "Antifragile" book and his Incerto work, Antifragile Learning is a knowledge generation system that improves with every error and mistake - via negativa!
In Section-4, we discuss the dangers of Scientism, which is what happens when capitalist interests mix with the pursuit of scientific research.
In Section-5, how ASB addresses the problems and challenges mentioned previously by creating an open source, distributed censorship-resistant platform and protocol, which is robust against misinformation and echo chambers, essentially becoming \emph{antifragile}---improving its knowledge quality through time with every new research added into the platform. 
In Section-6, we introduce a Technical Section that describes how a platform like ASB could be developed.
In the final Section, the finish by reflecting on the challenges ahead.