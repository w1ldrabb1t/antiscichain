\subsection{Consensus is Irrelevant (and potentially misleading!)}
It is tempting to think that if a majority agrees on something, it must be true. This notion underpins many democratic or crowd-based approaches to evaluating information, from social media ``likes'' to proposals of scientific DAOs where token-holders vote on the validity of research. However, scientific history teaches us that truth is not a popularity contest. A hypothesis can be widely believed and still be false (as in the geocentric model of the cosmos before Copernicus), or widely doubted and yet true (as in the case of meteorites, which were dismissed as superstition until evidence proved rocks do fall from the sky). Consensus can be a lagging indicator of truth: eventually most scientists came to accept plate tectonics or quantum mechanics, but only after decisive evidence forced a paradigm shift against initial majority skepticism.

Moreover, consensus-based systems are vulnerable to social biases and strategic manipulation. Groupthink can cause communities to rally around appealing ideas and dismiss legitimate criticisms without proper examination. In a token-voting scenario, a well-funded interest group could buy votes to tilt the ``truth'' in their favor, irrespective of the actual evidence. Even without malicious actors, a crowd might upvote information that is easy to understand or aligns with their prior beliefs, rather than that which is rigorously validated. Online, this is evident when misleading claims go viral due to charismatic presentation or echo-chamber effects, despite being unsupported by facts.

The ASB protocol circumvents these issues by \textbf{basing credibility on evidence, not on the number of people who endorse the claim}. In ASB, it does not matter if 1000 people \emph{believe} a result; what matters is whether experiments and data back it up. A single well-executed refutation in ASB can outweigh a sea of unsubstantiated upvotes. This aligns with the Popperian view that one critical test carries more weight than any amount of agreement. By design, the ASB scoring mechanism cannot be directly gamed by popularity: to increase a claim's score, one must provide new supporting evidence that will itself be scrutinized by others; to decrease a claim's score, one must provide a credible refutation. In both cases, it's the quality of evidence that moves the needle, not the headcount of supporters.