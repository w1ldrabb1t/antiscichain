\subsection{Knowledge Via Negativa}
The problem with most scientific investigation is that it focus on finding what works and proving that it works by establishing causality. The reason why this is a problem is because, like Taleb writes in Antifragile \cite{Taleb2012}(p.303), ``we know a lot more what is wrong than what is right, or, phrased according to the fragile/robust classification, negative knowledge (what is wrong, what does not work) is more robust to error than positive knowledge (what is right, what works). So knowledge grows by subtraction much more than by addition - given that what we know to be wrong cannot turn out to be right, at least not easily.''. In other words, we advance understanding by identifying and removing falsehoods more than by piling up tentative truths. Each time an experiment reveals that a theory does \emph{not} hold in some condition, our overall knowledge is improved by elimination of error. What remains after surviving many such tests is a theory that is more robust.

Using this Via Negativa approach, mistakes and errors are welcomed because the overall knowledge base improves. Taleb writes ``This creates \emph{a separation between good and bad systems}. Good systems such as airlines are set up to have small errors, independent from each other - or, in effect, negatively correlated to each other, \emph{since mistakes lower the odds of future mistakes}.''\cite{Taleb2012}

The ASB protocol is explicitly designed to be antifragile in an epistemic sense. When a paper's claim is proven wrong (say a high-profile result is decisively refuted by new data), it is not a failure of the system but a strengthening event. The fragility score of that claim will drop, and any other claims heavily reliant on it will be reevaluated in light of the new evidence, thereby preventing further spread of the flawed result. 
Meanwhile, the act of refutation (the new evidence) is itself rewarded with a high credibility for exposing a false claim. The overall knowledge graph becomes healthier: false nodes are pruned or marked, and only those claims that endure repeated challenges maintain high credibility. 
Thus, ASB \emph{benefits from} failed experiments and negative results---it becomes more reliable as more hypotheses are tested and challenged. This property maps closely to antifragility: the system improves its epistemic robustness through the very process of being stressed by refutations.