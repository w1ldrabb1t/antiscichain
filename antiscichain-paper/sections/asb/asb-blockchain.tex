\subsection{The Critical Role of the Blockchain}
The use of the Blockchain was something that I thought of since day one because it enables some key ASB design principles to be possible out-of-the-box, without requiring the need to rely on fragile methods that depend on centralized technology and human-based decision-making.

\subsubsection{Radical Transparency}
I wanted to make sure that all the research papers being added into the ASB platform, and linked to support or refute other research papers, would be publicly available for everyone to see at all times. But it doesn't stop there. I'm talking about being transparent to the point of showing the code that manages the ASB platform, so if you are wondering why a paper has a bad fragility score (more on this in the Technical Section), you can read and validate the code that generates that score. The code is not just "open source" - you can literally see the code that is running on the Blockchain!

\subsubsection{Full Immutability}
Once something is added to the blockchain, it cannot be changed without compromising the entire chain, which is securely hashed and cryptographically linked. In other words: what happens on the blockchain, stays on the blockchain—unchanged, forever. This means it's virtually impossible for anyone to alter previously published research without invalidating the block that represents that research. Each block, along with its place in the chain, is hashed. Changing even a single block alters its hash, which breaks the chain's validity. As a result, the altered copy would be rejected by all other honest nodes in the network.

\subsubsection{Censorship Resistance}
Following the footsteps of the previous point, one of the most powerful features of the Blockchain is that it resists censorship by design. As long as a submission meets the protocol’s requirements, no entity—no matter how powerful—can block it. For ASB, this is essential. It ensures that research challenging dominant narratives, institutions, or industries still finds its place in the permanent scientific record.

\subsubsection{Trustless}
The previous point already hints at the reason why the Blockchain is \emph{trustless}. In the Blockchain, we don't rely on any central authority - we don't need to! We simply trust the code that is running on the Blockchain as well as the cryptographic mechanisms that are in place to protect its integrity.
There's no need to trust a bank, a government, or a single institution to verify what happened. The system verifies itself. Every transaction is validated by a decentralized network of nodes, and every block is time-stamped, hashed, and permanently recorded. Consensus algorithms like Proof of Work or Proof of Stake make sure that bad actors can’t game the system without enormous cost. Trust, in this context, isn't granted—it's replaced by math, code, and transparency. That's what makes it powerful. That’s what makes it trustless.

\subsubsection{Distributed}
Platforms that run on centralized servers can be targeted and taken down much more easily than a distributed network. That’s one of the core design principles of the Blockchain—distributed computing power and storage. On the Blockchain, there is no single server or centralized entity controlling the data or executing the code. Instead, everything is distributed across all the nodes in the network. Each node stores a full (or partial, depending on the implementation) copy of the ledger and participates in maintaining the system’s integrity. This distribution makes the network resilient, fault-tolerant, and far more difficult to compromise.

\subsubsection{Open}
The Blockchain embraces radical openness where anyone can use get access to it and use it. This is a key aspect that I wanted to bring into the ASB platform because I want to make this platform censor-free while at the same time making it impossible for anyone to hide from the mistakes they did in their past research (eg. impossible to hide iatrogenesis). The goal is to prevent any central gatekeeper from suppressing any kind of results. If a paper presents inconvenient evidence or challenges powerful interests, it should still be available on the ASB network for open access to everyone.