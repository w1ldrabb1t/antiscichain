\subsection{Please break me - if you can}
At the core of the ASB epistemology is Karl Popper's notion of \emph{falsifiability}\cite{Popper1963}. Popper argued that for a hypothesis to be scientific, it must be testable in a way that could potentially show it to be false. In his words, ``Every genuine test of a theory is an attempt to falsify it, or refute it''\cite{Popper1963}. 

\textbf{Confirming instances of a theory can increase our confidence in it, but can never conclusively prove it true}. By contrast, a single counter-example can definitively show a universal hypothesis to be false. Albert Einstein echoed this asymmetry: \emph{``No amount of experimentation can ever prove me right; a single experiment can prove me wrong.''}. The scientific method, ideally, is a process of bold conjectures and rigorous attempts at refutation.

In line with this philosophy, the ASB protocol treats \textbf{negative evidence} (refutations, failed replications, contradictions) as more decisive than \textbf{positive evidence} (confirmations or replications). The discovery of a ``black swan''---an outcome that contradicts a prevailing hypothesis---is weighted more heavily in the research fragility score than dozens of observations consistent with the hypothesis. This is not to downplay the value of replication, which is crucial for verifying robustness, but to recognize that from an epistemic standpoint, refutation provides a much stronger signal that confirmation does not. 
By structuring the scoring to heavily reward successful falsification attempts, ASB incentivizes researchers to actively test the boundaries of current knowledge. Rather than accumulate votes of confidence, a claim must survive attempts to dismantle it.

It's like labeling a package with the intriguing message - "please break me, if you can". If the package is fragile, then it will break and be removed from the area (while others look at you wondering what's wrong with you). If the package is robust, it will survive your attempts to break it (while you might hurt yourself in the process). Finally, if the package is antifragile, then you will only make it stronger with your attempts to break it. The same is true for the research nodes in the ASB network.
