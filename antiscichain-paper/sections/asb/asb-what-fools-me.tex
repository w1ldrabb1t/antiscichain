\subsection{What fools me, makes us all smarter (and antifragile)}
The combination of the evidence-driven scoring and open access infrastructure makes ASB a \textbf{self-correcting system}. When misinformation or erroneous results enter the system (intentionally or not), they might at first get some traction if they come with what appears to be supporting evidence. However, because the system rewards attempts at refutation, there is a strong incentive for other researchers to scrutinize and test such claims. If the claim is false, eventually someone will produce the refuting data. When that happens, ASB will adjust the scores accordingly: the incorrect claim's credibility drops, and the new evidence is recorded for all to see. Far from being a weakness, these failures are how the system learns.

This is the essence of antifragility in the epistemic realm. The more we probe and challenge the body of knowledge, the stronger the overall confidence in what remains. Each correction not only fixes a particular false claim, but also increases trust in the mechanism of science itself. In ASB, this process is accelerated and made explicit: every refutation immediately quantifies its impact on research fragility scores, and the network graph highlights which other results might be affected. The system thus \emph{benefits from} the very volatility of research outcomes that might trouble a consensus-based approach. Where a consensus system might resist or delay acknowledging a critical refutation (since it must sway the opinions of the majority), ASB incorporates it algorithmically as soon as it is available.

Furthermore, because ASB does not converge on a \emph{declared truth} but always remains open to new evidence, it resists ossification. Even widely accepted theories continue to have high ASB scores not because everyone agrees per se, but because they have accumulated a long chain of supporting evidence and no serious refutations. Should new information arise that challenges them, the system will register it. This keeps science perpetually open to revision, which is exactly as it should be.

In terms of misinformation attacks (say, a coordinated attempt to flood the system with fake supporting studies for a false claim), ASB has intrinsic defenses. Low-quality or fraudulent studies typically will not withstand scrutiny—other scientists can attempt to replicate them or find inconsistencies. As those fraudulent nodes get refuted, their malfeasance is exposed and their contributions to supporting the false claim are nullified by stronger contrary evidence. An attack would have to not only inject false information but also somehow prevent any refutations or alternative data from emerging, which is implausible in a globally open network of experts. In effect, attempts to game the ASB only create more opportunities for diligent researchers to debunk and strengthen the corpus. The outcome is a system that, like a biological immune system, identifies and neutralizes false signals and becomes more resilient to them in the future.

Overall, the ASB protocol provides a framework for scientific knowledge that is more robust to human biases and social dynamics. It re-focuses validation on empirical reality as it unfolds through experiments. By doing so in a decentralized and transparent way, it aligns closely with the ideal of science as a self-correcting process. It offers a path to \emph{truth without consensus}—allowing consensus to be a result of truth, not a substitute for it.
