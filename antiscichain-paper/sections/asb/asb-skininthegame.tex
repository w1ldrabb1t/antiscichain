\subsection{Skin in the Game for the Research}
Another concept from Taleb's Incerto is the idea of \emph{skin in the game}\cite{Taleb2018}. In domains of uncertainty, Taleb emphasizes that having "skin in the game"—being exposed to the cost of being wrong—is not only an ethical requirement (one should never be antifragile by offloading the downside of their mistakes onto others), but also a way to prevent clueless yet powerful actors from tampering with complex, non-linear systems (and blowing us all up in the process).
However, \emph{skin in the game} should not be confused with measuring personal credibility. Someone might have a long history of being wrong and still produce valid, even groundbreaking, research. That’s why the ASB platform doesn't track or score the credibility of authors — we don't care about that. What we \emph{do} care about is the research itself. 
This is \emph{skin in the game} applied to the research object, not the researcher. We focus on what a paper confirms or refutes, not who wrote it. In ASB, every claim must stand or fall on its own evidence and fragility score—regardless of reputation.